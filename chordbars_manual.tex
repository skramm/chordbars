\documentclass[11pt]{article}
\usepackage[utf8]{inputenc}
\usepackage[left=2.00cm,vmargin=1cm]{geometry}

\usepackage{listings}
\lstset{
	frame=single
	,language=TeX
	,basicstyle=\ttfamily,
}
\usepackage{chordbars}

\title{Package chordbars manual}
\author{S. Kramm}

\begin{document}
\maketitle

\begin{abstract}
This Tikz-based package can be useful for guitar / bass / piano / whatever players that are playing musical accompaniment.
It produces rectangular song patterns with "1 square per bar", with the chord shown inside the square.
It also handles the song structure by showing the bar count.
\end{abstract}

\section{Usage}
This package provides a single environnment named {\ttfamily chordbar}, inside which one can define the chords for each bar.

For example, this code:

\begin{lstlisting}
\begin{chordbar}{4}{1}{Intro}
\chordf{1}{C}
\chordf{2}{D}
\chordh{3}{Em}{1}
\chordh{3}{B}{2}
\chordf{4}{Em}
\end{chordbar}
\end{lstlisting}

will produce this:

\begin{chordbar}{4}{1}{Intro}
\chordf{1}{C}
\chordf{2}{D}
\chordh{3}{Em}{1}
\chordh{3}{B}{2}
\chordf{4}{Em}
\end{chordbar}


This environment has three arguments:
\begin{enumerate}
\item the number of bars needed for a pattern,
\item the number of times the whole pattern gets repeated (useful for bars counting),
\item the part name.
\end{enumerate}

It also keep tracks of the measure count, including repeatitions. For example, this:

\begin{lstlisting}
\resetchordbars
\begin{chordbar}{4}{3}{part1}
\end{chordbar}
\begin{chordbar}{4}{1}{part 2}
\end{chordbar}
\end{lstlisting}

will give the following output, correctly printing out '13' as the initial bar number of the second part (part 1 is 4 bars long and is repeated 3 times).

\resetchordbars
\begin{chordbar}{4}{3}{}
\end{chordbar}

\begin{chordbar}{4}{1}{}
\end{chordbar}


\section{FAQ}

\begin{itemize}
\item Q: How do I do with bars having more/less than 4 beats ? \\
A: This has not been considered here.
\end{itemize}
\end{document}

% this is mostly a test file, to check features

\documentclass[11pt]{article}
\usepackage[utf8]{inputenc}
\usepackage[left=2.00cm,vmargin=1cm]{geometry}

\usepackage{chordbars}

\title{Song Sample with a really realllly long name, just to test}
\author{Author name}


\begin{document}
%\bpm{120}
%\bpbthree


\countbarsYes

\songtitle

%\def\barsize{1.5}%	
This is an example done with the chordbars package.
It is just a pseudo-demo of what the package can achieve, don't take it too seriously.



\begin{chordbar}{4}{Intro}
\chordf{C\flat}
%\chordf{C\flat}
\chordh{A#}{D\sharp}
%\repeatBarPair
\end{chordbar}


Whole line of same chord:

\chordline{C#m7}{8}{Intro}

You can print some notes between the different parts

\def\barsize{1.8}%	
\def\chordFontSize{\huge\bfseries}
\begin{chordbar}[3]{6}{Intro b}
\chordf{C#}
\repeatBar
\chordh{A}{D}
\chordf{F}
\chordf{F}
\chordf{D\MajS}
\end{chordbar}

%\begin{chordbar}{4}{1}{Intro c}
%\chordfn{1}{C}
%%\chordfn{2}{D}
%\chordfn{2}{Em}
%\repeatchord
%\repeatchord
%\end{chordbar}

\def\barsize{1.4}%
% this part has 8 bars (4 bars, 2 lines)
\begin{chordbar}[2]{4}{TEST}
\chordf{C}
\chordf{D}
\chordh{E}{F}
\repeatBar
\end{chordbar}

\def\barsize{2}
\def\chordFontSize{\Large\bfseries}
\begin{chordbar}{5}{Test of repeatbarpair}
\chordh{G}{C7}
\chordf{F\MajS}
\repeatBarPair
\chordf{F\MajS}
\addHalfBar{Dm}
\end{chordbar}


\begin{chordbar}{4}{Test of addHalfBar}
\chordh{G}{C7}
\chordf{FMajS}
\repeatBar
\chordf{Dm}
\addHalfBar{Gm}
\end{chordbar}


\begin{chordbar}[3]{6}{6 bars on 2 lines, repeated 3 times}
\chordf{C}
\chordf{D}
\chordh{E}{F}
\repeatBar
\chordf{D}
\repeatBar
\end{chordbar}

\begin{chordbar}[4]{11}{11 bars repeated 4 times, with an added half-bar}
\chordf{C}
\chordh{D}{G}
\chordf{E}
%\chordf{F}
\newchordline
\chordf{G\#}
\repeatBar
\repeatBar
\repeatBar
\chordf{C\susF}
\repeatBar
\repeatBar
\addHalfBar{Bb}
\end{chordbar}

\def\NumberOfBarsPerLine{5}

\begin{chordbar}[2]{10}{2 lines of 5 bars, repeated twice}
\chordf{C}
\chordf{D}
\chordf{E}
\chordf{F}
\newchordline
\chordf{G#}
\chordh{A}{G#}
\repeatBar
\end{chordbar}


\def\NumberOfBarsPerLine{8}
\begin{chordbar}[2]{6}{coda}
\chordf{E}
\chordh{A}{F}
\end{chordbar}


\printNbBars

\end{document}
